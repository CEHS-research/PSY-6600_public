% Options for packages loaded elsewhere
\PassOptionsToPackage{unicode}{hyperref}
\PassOptionsToPackage{hyphens}{url}
%
\documentclass[
]{article}
\usepackage{lmodern}
\usepackage{amsmath}
\usepackage{ifxetex,ifluatex}
\ifnum 0\ifxetex 1\fi\ifluatex 1\fi=0 % if pdftex
  \usepackage[T1]{fontenc}
  \usepackage[utf8]{inputenc}
  \usepackage{textcomp} % provide euro and other symbols
  \usepackage{amssymb}
\else % if luatex or xetex
  \usepackage{unicode-math}
  \defaultfontfeatures{Scale=MatchLowercase}
  \defaultfontfeatures[\rmfamily]{Ligatures=TeX,Scale=1}
\fi
% Use upquote if available, for straight quotes in verbatim environments
\IfFileExists{upquote.sty}{\usepackage{upquote}}{}
\IfFileExists{microtype.sty}{% use microtype if available
  \usepackage[]{microtype}
  \UseMicrotypeSet[protrusion]{basicmath} % disable protrusion for tt fonts
}{}
\makeatletter
\@ifundefined{KOMAClassName}{% if non-KOMA class
  \IfFileExists{parskip.sty}{%
    \usepackage{parskip}
  }{% else
    \setlength{\parindent}{0pt}
    \setlength{\parskip}{6pt plus 2pt minus 1pt}}
}{% if KOMA class
  \KOMAoptions{parskip=half}}
\makeatother
\usepackage{xcolor}
\IfFileExists{xurl.sty}{\usepackage{xurl}}{} % add URL line breaks if available
\IfFileExists{bookmark.sty}{\usepackage{bookmark}}{\usepackage{hyperref}}
\hypersetup{
  pdftitle={Psy/Educ 6600: Categorical Data Analysis},
  pdfauthor={Your Name},
  hidelinks,
  pdfcreator={LaTeX via pandoc}}
\urlstyle{same} % disable monospaced font for URLs
\usepackage[margin=1in]{geometry}
\usepackage{color}
\usepackage{fancyvrb}
\newcommand{\VerbBar}{|}
\newcommand{\VERB}{\Verb[commandchars=\\\{\}]}
\DefineVerbatimEnvironment{Highlighting}{Verbatim}{commandchars=\\\{\}}
% Add ',fontsize=\small' for more characters per line
\usepackage{framed}
\definecolor{shadecolor}{RGB}{248,248,248}
\newenvironment{Shaded}{\begin{snugshade}}{\end{snugshade}}
\newcommand{\AlertTok}[1]{\textcolor[rgb]{0.94,0.16,0.16}{#1}}
\newcommand{\AnnotationTok}[1]{\textcolor[rgb]{0.56,0.35,0.01}{\textbf{\textit{#1}}}}
\newcommand{\AttributeTok}[1]{\textcolor[rgb]{0.77,0.63,0.00}{#1}}
\newcommand{\BaseNTok}[1]{\textcolor[rgb]{0.00,0.00,0.81}{#1}}
\newcommand{\BuiltInTok}[1]{#1}
\newcommand{\CharTok}[1]{\textcolor[rgb]{0.31,0.60,0.02}{#1}}
\newcommand{\CommentTok}[1]{\textcolor[rgb]{0.56,0.35,0.01}{\textit{#1}}}
\newcommand{\CommentVarTok}[1]{\textcolor[rgb]{0.56,0.35,0.01}{\textbf{\textit{#1}}}}
\newcommand{\ConstantTok}[1]{\textcolor[rgb]{0.00,0.00,0.00}{#1}}
\newcommand{\ControlFlowTok}[1]{\textcolor[rgb]{0.13,0.29,0.53}{\textbf{#1}}}
\newcommand{\DataTypeTok}[1]{\textcolor[rgb]{0.13,0.29,0.53}{#1}}
\newcommand{\DecValTok}[1]{\textcolor[rgb]{0.00,0.00,0.81}{#1}}
\newcommand{\DocumentationTok}[1]{\textcolor[rgb]{0.56,0.35,0.01}{\textbf{\textit{#1}}}}
\newcommand{\ErrorTok}[1]{\textcolor[rgb]{0.64,0.00,0.00}{\textbf{#1}}}
\newcommand{\ExtensionTok}[1]{#1}
\newcommand{\FloatTok}[1]{\textcolor[rgb]{0.00,0.00,0.81}{#1}}
\newcommand{\FunctionTok}[1]{\textcolor[rgb]{0.00,0.00,0.00}{#1}}
\newcommand{\ImportTok}[1]{#1}
\newcommand{\InformationTok}[1]{\textcolor[rgb]{0.56,0.35,0.01}{\textbf{\textit{#1}}}}
\newcommand{\KeywordTok}[1]{\textcolor[rgb]{0.13,0.29,0.53}{\textbf{#1}}}
\newcommand{\NormalTok}[1]{#1}
\newcommand{\OperatorTok}[1]{\textcolor[rgb]{0.81,0.36,0.00}{\textbf{#1}}}
\newcommand{\OtherTok}[1]{\textcolor[rgb]{0.56,0.35,0.01}{#1}}
\newcommand{\PreprocessorTok}[1]{\textcolor[rgb]{0.56,0.35,0.01}{\textit{#1}}}
\newcommand{\RegionMarkerTok}[1]{#1}
\newcommand{\SpecialCharTok}[1]{\textcolor[rgb]{0.00,0.00,0.00}{#1}}
\newcommand{\SpecialStringTok}[1]{\textcolor[rgb]{0.31,0.60,0.02}{#1}}
\newcommand{\StringTok}[1]{\textcolor[rgb]{0.31,0.60,0.02}{#1}}
\newcommand{\VariableTok}[1]{\textcolor[rgb]{0.00,0.00,0.00}{#1}}
\newcommand{\VerbatimStringTok}[1]{\textcolor[rgb]{0.31,0.60,0.02}{#1}}
\newcommand{\WarningTok}[1]{\textcolor[rgb]{0.56,0.35,0.01}{\textbf{\textit{#1}}}}
\usepackage{longtable,booktabs}
\usepackage{calc} % for calculating minipage widths
% Correct order of tables after \paragraph or \subparagraph
\usepackage{etoolbox}
\makeatletter
\patchcmd\longtable{\par}{\if@noskipsec\mbox{}\fi\par}{}{}
\makeatother
% Allow footnotes in longtable head/foot
\IfFileExists{footnotehyper.sty}{\usepackage{footnotehyper}}{\usepackage{footnote}}
\makesavenoteenv{longtable}
\usepackage{graphicx}
\makeatletter
\def\maxwidth{\ifdim\Gin@nat@width>\linewidth\linewidth\else\Gin@nat@width\fi}
\def\maxheight{\ifdim\Gin@nat@height>\textheight\textheight\else\Gin@nat@height\fi}
\makeatother
% Scale images if necessary, so that they will not overflow the page
% margins by default, and it is still possible to overwrite the defaults
% using explicit options in \includegraphics[width, height, ...]{}
\setkeys{Gin}{width=\maxwidth,height=\maxheight,keepaspectratio}
% Set default figure placement to htbp
\makeatletter
\def\fps@figure{htbp}
\makeatother
\usepackage[normalem]{ulem}
% Avoid problems with \sout in headers with hyperref
\pdfstringdefDisableCommands{\renewcommand{\sout}{}}
\setlength{\emergencystretch}{3em} % prevent overfull lines
\providecommand{\tightlist}{%
  \setlength{\itemsep}{0pt}\setlength{\parskip}{0pt}}
\setcounter{secnumdepth}{-\maxdimen} % remove section numbering
\usepackage{rotating}
\usepackage{graphicx}
\usepackage{booktabs}
\ifluatex
  \usepackage{selnolig}  % disable illegal ligatures
\fi

\title{Psy/Educ 6600: Categorical Data Analysis}
\usepackage{etoolbox}
\makeatletter
\providecommand{\subtitle}[1]{% add subtitle to \maketitle
  \apptocmd{\@title}{\par {\large #1 \par}}{}{}
}
\makeatother
\subtitle{Chapter 20: Chi Squared TEsts}
\author{Your Name}
\date{Spring 2020}

\begin{document}
\maketitle

{
\setcounter{tocdepth}{3}
\tableofcontents
}
\clearpage

\listoftables
\listoffigures

\clearpage

\hypertarget{preparation}{%
\section{PREPARATION}\label{preparation}}

\hypertarget{packages}{%
\subsection{Packages}\label{packages}}

Make sure the packages are \textbf{installed} \emph{(Package tab)}

\begin{Shaded}
\begin{Highlighting}[]
\FunctionTok{library}\NormalTok{(readxl)}
\FunctionTok{library}\NormalTok{(magrittr)     }\CommentTok{\# Forward pipes in R}
\FunctionTok{library}\NormalTok{(tidyverse)    }\CommentTok{\# Loads several very helpful \textquotesingle{}tidy\textquotesingle{} packages}
\FunctionTok{library}\NormalTok{(furniture)    }\CommentTok{\# Nice tables (by our own Tyson Barrett)}
\FunctionTok{library}\NormalTok{(effectsize)   }\CommentTok{\# effect size calculation}
\end{Highlighting}
\end{Shaded}

\clearpage

\hypertarget{section-a}{%
\section{SECTION A}\label{section-a}}

\hypertarget{datasets}{%
\subsection{Datasets}\label{datasets}}

\begin{Shaded}
\begin{Highlighting}[]
\NormalTok{table\_soda }\OtherTok{\textless{}{-}} \FunctionTok{c}\NormalTok{(}\AttributeTok{X =} \DecValTok{27}\NormalTok{, }
                \AttributeTok{Y =} \DecValTok{15}\NormalTok{, }
                \AttributeTok{Z =} \DecValTok{24}\NormalTok{) }\SpecialCharTok{\%\textgreater{}\%} 
  \FunctionTok{as.table}\NormalTok{()}

\NormalTok{table\_season }\OtherTok{\textless{}{-}} \FunctionTok{c}\NormalTok{(}\AttributeTok{spring =} \DecValTok{30}\NormalTok{, }
                  \AttributeTok{summer =} \DecValTok{40}\NormalTok{, }
                  \AttributeTok{fall   =} \DecValTok{20}\NormalTok{,}
                  \AttributeTok{winter =} \DecValTok{10}\NormalTok{) }\SpecialCharTok{\%\textgreater{}\%} 
  \FunctionTok{as.table}\NormalTok{()}

\NormalTok{table\_dx }\OtherTok{\textless{}{-}} \FunctionTok{c}\NormalTok{(}\AttributeTok{Schizophrenic =} \DecValTok{60}\NormalTok{, }
              \AttributeTok{Depressed     =} \DecValTok{30}\NormalTok{,}
              \AttributeTok{Bipolar       =} \DecValTok{10}\NormalTok{) }\SpecialCharTok{\%\textgreater{}\%} 
  \FunctionTok{as.table}\NormalTok{()}
\end{Highlighting}
\end{Shaded}

\clearpage

\hypertarget{table_soda-blind-taste-test-of-soft-drinks-given-counts}{%
\subsection{\texorpdfstring{\texttt{table\_soda} Blind taste test of
soft drinks (given
counts)}{table\_soda Blind taste test of soft drinks (given counts)}}\label{table_soda-blind-taste-test-of-soft-drinks-given-counts}}

\hypertarget{a-3-1-way-chi-squared-test---goodness-of-fit-equally-likely}{%
\subsubsection{20A 3: 1-way Chi-squared Test - Goodness-of-Fit (equally
likely)}\label{a-3-1-way-chi-squared-test---goodness-of-fit-equally-likely}}

\textbf{TEXTBOOK QUESTION:} \emph{A soft drink manufacturer is
conducting a blind taste test to compare its best-selling product
(\texttt{X}) with two leading competitors (\texttt{Y} and \texttt{Z}).
Each subject tastes all three and selects the one that tastes best to
him or her. (a) What is the appropriate null hypothesis for this study?
(b) If 27 subjects prefer product \texttt{X}, 15 prefer product
\texttt{Y}, and 24 prefer product \texttt{Z}, can you reject the null
hypothesis at the .05 level?}

\begin{center}\rule{0.5\linewidth}{0.5pt}\end{center}

\textbf{DIRECTIONS:} Use the \texttt{chisq.test()} function to perform a
Goodnes-of-Fit or one-way Chi-Squared test to see if the observed counts
(\texttt{table\_soda}) are significantly different from being equally
distributed among the three soft drinks. Save the fitted model as
\texttt{chisq\_soda}.

\begin{quote}
\textbf{NOTE:} You do not need to declare any options inside the
\texttt{chisq.test()} function, as the default is to use equally likely
probabilities.
\end{quote}

\begin{Shaded}
\begin{Highlighting}[]
\CommentTok{\# Run the 1{-}way chi{-}square test for equally likely}
\end{Highlighting}
\end{Shaded}

\textbf{DIRECTIONS:} Folow the tutorial to create a table comparing the
observed and expected counts.

\begin{quote}
\textbf{HINT} You may \emph{copy-and-paste} the code from the chunked
named \texttt{tutorial\_chiSq\_GoF\_EL\_counts}, but remember to change
the name of the model (appears before the \$-sign in two places).
\end{quote}

\begin{Shaded}
\begin{Highlighting}[]
\CommentTok{\# Request the observed and expected counts}
\end{Highlighting}
\end{Shaded}

\textbf{DIRECTIONS:} Place the model's name (\texttt{chisq\_soda}) in
the following chunck, so that when run it will display the full output
of the Chi-squared test.

\begin{Shaded}
\begin{Highlighting}[]
\CommentTok{\# Diplay the full output}
\end{Highlighting}
\end{Shaded}

\clearpage

\hypertarget{table_season-psychiatric-hospital-admits-by-season}{%
\subsection{\texorpdfstring{\texttt{table\_season} Psychiatric Hospital
Admits by
Season}{table\_season Psychiatric Hospital Admits by Season}}\label{table_season-psychiatric-hospital-admits-by-season}}

\hypertarget{a-7-1-way-chi-squared-test---goodness-of-fit-equally-likely}{%
\subsubsection{20A 7: 1-way Chi-squared Test - Goodness-of-Fit (equally
likely)}\label{a-7-1-way-chi-squared-test---goodness-of-fit-equally-likely}}

\textbf{TEXTBOOK QUESTION:} \emph{It has been suggested that admissions
to psychiatric hospitals may vary by season. One hypothetical hospital
admitted 100 patients last year: 30 in the spring; 40 in the summer; 20
in the fall; and 10 in the winter. Use the chi-square test to evaluate
the hypothesis that mental illness emergencies are evenly distributed
throughout the year.}

\begin{center}\rule{0.5\linewidth}{0.5pt}\end{center}

\textbf{DIRECTIONS:} Use the \texttt{chisq.test()} function to perform a
Goodnes-of-Fit or one-way Chi-Squared test to see if the observed counts
(\texttt{table\_season}) are significantly different from being equally
distributed among the four seasons. Save the fitted model as
\texttt{chisq\_season}.

\begin{quote}
\textbf{NOTE:} You do not need to declare any options inside the
\texttt{chisq.test()} function, as the default is to use equally likely
probabilities.
\end{quote}

\begin{Shaded}
\begin{Highlighting}[]
\CommentTok{\# Run the 1{-}way chi{-}square test for equally likely}
\end{Highlighting}
\end{Shaded}

\textbf{DIRECTIONS:} Folow the tutorial to create a table comparing the
observed and expected counts.

\begin{quote}
\textbf{HINT} You may \emph{copy-and-paste} the code from the chunked
named \texttt{tutorial\_chiSq\_GoF\_EL\_counts}, but remember to change
the name of the model (appears before the \$-sign in two places).
\end{quote}

\begin{Shaded}
\begin{Highlighting}[]
\CommentTok{\# Request the observed and expected counts}
\end{Highlighting}
\end{Shaded}

\textbf{DIRECTIONS:} Place the model's name (\texttt{chisq\_season}) in
the following chunck, so that when run it will display the full output
of the Chi-squared test.

\begin{Shaded}
\begin{Highlighting}[]
\CommentTok{\# Diplay the full output}
\end{Highlighting}
\end{Shaded}

\clearpage

\hypertarget{table_dx-psychiatric-hospital-admits-by-diagnosis}{%
\subsection{\texorpdfstring{\texttt{table\_dx} Psychiatric Hospital
Admits by
Diagnosis}{table\_dx Psychiatric Hospital Admits by Diagnosis}}\label{table_dx-psychiatric-hospital-admits-by-diagnosis}}

\hypertarget{a-8-1-way-chi-squared-test---goodness-of-fit-hypothesised-probabilities}{%
\subsubsection{20A 8: 1-way Chi-squared Test - Goodness-of-Fit
(hypothesised
probabilities)}\label{a-8-1-way-chi-squared-test---goodness-of-fit-hypothesised-probabilities}}

\textbf{TEXTBOOK QUESTION:} \emph{Of the 100 psychiatric patients
referred to in the previous exercise, 60 were diagnosed as
schizophrenic, 30 were severely depressed, and 10 had a bipolar
disorder. Assuming that the national percentages for psychiatric
admissions are 55\% schizophrenic, 39\% depressive, and 6\% bipolar, use
the chi-square test to evaluate the null hypothesis that this particular
hospital is receiving a random selection of psychiatric patients from
the national population.}

\begin{center}\rule{0.5\linewidth}{0.5pt}\end{center}

\textbf{DIRECTIONS:} Use the \texttt{chisq.test()} function to perform a
Goodnes-of-Fit or one-way Chi-Squared test to see if the observed counts
(\texttt{table\_dx}) are significantly different from being equally
distributed among the three diagnoses. Save the fitted model as
\texttt{chisq\_dx}.

\begin{quote}
\textbf{NOTE:} You \textbf{DO} need to declare the probabilities, as the
default is to use equally likely probabilities. You may do this by
including \texttt{p\ =\ c(.55,\ .39,\ .06)} within the
\texttt{chisq.test()} function.
\end{quote}

\begin{Shaded}
\begin{Highlighting}[]
\CommentTok{\# Run the 1{-}way chi{-}square test for hypothesized probabilities}
\end{Highlighting}
\end{Shaded}

\textbf{DIRECTIONS:} Folow the tutorial to create a table comparing the
observed and expected counts.

\begin{quote}
\textbf{HINT} You may \emph{copy-and-paste} the code from the chunked
named \texttt{tutorial\_chiSq\_GoF\_EL\_counts}, but remember to change
the name of the model (appears before the \$-sign in two places).
\end{quote}

\begin{Shaded}
\begin{Highlighting}[]
\CommentTok{\# Request the observed and expected counts}
\end{Highlighting}
\end{Shaded}

\textbf{DIRECTIONS:} Place the model's name (\texttt{chisq\_rx}) in the
following chunck, so that when run it will display the full output of
the Chi-squared test.

\begin{Shaded}
\begin{Highlighting}[]
\CommentTok{\# Diplay the full output}
\end{Highlighting}
\end{Shaded}

\clearpage

\hypertarget{section-b}{%
\section{Section B}\label{section-b}}

\hypertarget{datasets-1}{%
\subsection{Datasets}\label{datasets-1}}

\begin{Shaded}
\begin{Highlighting}[]
\NormalTok{react\_wealth }\OtherTok{\textless{}{-}} \FunctionTok{data.frame}\NormalTok{(}\AttributeTok{poor    =} \FunctionTok{c}\NormalTok{(}\DecValTok{16}\NormalTok{, }\DecValTok{8}\NormalTok{,  }\DecValTok{6}\NormalTok{),}
                           \AttributeTok{middle  =} \FunctionTok{c}\NormalTok{(}\DecValTok{10}\NormalTok{, }\DecValTok{6}\NormalTok{, }\DecValTok{14}\NormalTok{),}
                           \AttributeTok{wealthy =} \FunctionTok{c}\NormalTok{( }\DecValTok{7}\NormalTok{, }\DecValTok{5}\NormalTok{, }\DecValTok{18}\NormalTok{),}
                           \AttributeTok{row.names =} \FunctionTok{c}\NormalTok{(}\StringTok{"ignores"}\NormalTok{, }\StringTok{"talks"}\NormalTok{, }\StringTok{"helps"}\NormalTok{)) }\SpecialCharTok{\%\textgreater{}\%} 
  \FunctionTok{as.matrix}\NormalTok{() }\SpecialCharTok{\%\textgreater{}\%} 
  \FunctionTok{as.table}\NormalTok{() }


\NormalTok{speed\_voice }\OtherTok{\textless{}{-}} \FunctionTok{data.frame}\NormalTok{(}\AttributeTok{child =} \FunctionTok{c}\NormalTok{(}\DecValTok{5}\NormalTok{, }\DecValTok{2}\NormalTok{),}
                          \AttributeTok{woman =} \FunctionTok{c}\NormalTok{(}\DecValTok{3}\NormalTok{, }\DecValTok{4}\NormalTok{),}
                          \AttributeTok{man   =} \FunctionTok{c}\NormalTok{(}\DecValTok{1}\NormalTok{, }\DecValTok{6}\NormalTok{),}
                          \AttributeTok{row.names =} \FunctionTok{c}\NormalTok{(}\StringTok{"fast"}\NormalTok{, }\StringTok{"slow"}\NormalTok{)) }\SpecialCharTok{\%\textgreater{}\%} 
  \FunctionTok{as.matrix}\NormalTok{() }\SpecialCharTok{\%\textgreater{}\%} 
  \FunctionTok{as.table}\NormalTok{() }


\NormalTok{data\_12b4 }\OtherTok{\textless{}{-}} \FunctionTok{data.frame}\NormalTok{(}\AttributeTok{child =} \FunctionTok{c}\NormalTok{(}\DecValTok{10}\NormalTok{, }\DecValTok{12}\NormalTok{, }\DecValTok{15}\NormalTok{, }\DecValTok{11}\NormalTok{,  }\DecValTok{5}\NormalTok{,  }\DecValTok{7}\NormalTok{,  }\DecValTok{2}\NormalTok{),}
                        \AttributeTok{woman =} \FunctionTok{c}\NormalTok{(}\DecValTok{17}\NormalTok{, }\DecValTok{13}\NormalTok{, }\DecValTok{16}\NormalTok{, }\DecValTok{12}\NormalTok{,  }\DecValTok{7}\NormalTok{,  }\DecValTok{8}\NormalTok{,  }\DecValTok{3}\NormalTok{),}
                        \AttributeTok{man   =} \FunctionTok{c}\NormalTok{(}\DecValTok{20}\NormalTok{, }\DecValTok{25}\NormalTok{, }\DecValTok{14}\NormalTok{, }\DecValTok{17}\NormalTok{, }\DecValTok{12}\NormalTok{, }\DecValTok{18}\NormalTok{,  }\DecValTok{7}\NormalTok{))}
\end{Highlighting}
\end{Shaded}

\clearpage

\hypertarget{react_wealth-reaction-to-wealth}{%
\subsection{\texorpdfstring{\texttt{react\_wealth} Reaction to
Wealth}{react\_wealth Reaction to Wealth}}\label{react_wealth-reaction-to-wealth}}

\hypertarget{b-4-2-way-chi-squared-test---independence}{%
\subsubsection{20B 4: 2-way Chi-squared Test -
Independence}\label{b-4-2-way-chi-squared-test---independence}}

\textbf{TEXTBOOK QUESTION:} \emph{A social psychologist is studying
whether people are more likely to help a poor person or a rich person
who they find lying on the floor. The three conditions all involve an
elderly woman who falls down in a shopping mall (when only one person at
a time is nearby). The independent variable concerns the apparent wealth
of the woman; she is dressed to appear either poor, wealthy, or middle
class. The reaction of each bystander is classified in one of three
ways: ignoring her, asking if she is all right, and helping her to her
feet. The data appear in the contingency table below. (a) Test the null
hypothesis at the .01 level. Is there evidence for an association
between the apparent wealth of the victim and the amount of help
provided by a bystander? \sout{(b) Calculate Cramer's phi for these
data. What can you say about the strength of the relationship between
the two variables?}}

\begin{Shaded}
\begin{Highlighting}[]
\CommentTok{\# Display the observed counts}
\NormalTok{react\_wealth }\SpecialCharTok{\%\textgreater{}\%} 
  \FunctionTok{addmargins}\NormalTok{() }\SpecialCharTok{\%\textgreater{}\%} 
\NormalTok{  pander}\SpecialCharTok{::}\FunctionTok{pander}\NormalTok{()}
\end{Highlighting}
\end{Shaded}

\begin{longtable}[]{@{}ccccc@{}}
\toprule
\begin{minipage}[b]{(\columnwidth - 4\tabcolsep) * \real{0.19}}\centering
~\strut
\end{minipage} &
\begin{minipage}[b]{(\columnwidth - 4\tabcolsep) * \real{0.10}}\centering
poor\strut
\end{minipage} &
\begin{minipage}[b]{(\columnwidth - 4\tabcolsep) * \real{0.12}}\centering
middle\strut
\end{minipage} &
\begin{minipage}[b]{(\columnwidth - 4\tabcolsep) * \real{0.14}}\centering
wealthy\strut
\end{minipage} &
\begin{minipage}[b]{(\columnwidth - 4\tabcolsep) * \real{0.08}}\centering
Sum\strut
\end{minipage}\tabularnewline
\midrule
\endhead
\begin{minipage}[t]{(\columnwidth - 4\tabcolsep) * \real{0.19}}\centering
\textbf{ignores}\strut
\end{minipage} &
\begin{minipage}[t]{(\columnwidth - 4\tabcolsep) * \real{0.10}}\centering
16\strut
\end{minipage} &
\begin{minipage}[t]{(\columnwidth - 4\tabcolsep) * \real{0.12}}\centering
10\strut
\end{minipage} &
\begin{minipage}[t]{(\columnwidth - 4\tabcolsep) * \real{0.14}}\centering
7\strut
\end{minipage} &
\begin{minipage}[t]{(\columnwidth - 4\tabcolsep) * \real{0.08}}\centering
33\strut
\end{minipage}\tabularnewline
\begin{minipage}[t]{(\columnwidth - 4\tabcolsep) * \real{0.19}}\centering
\textbf{talks}\strut
\end{minipage} &
\begin{minipage}[t]{(\columnwidth - 4\tabcolsep) * \real{0.10}}\centering
8\strut
\end{minipage} &
\begin{minipage}[t]{(\columnwidth - 4\tabcolsep) * \real{0.12}}\centering
6\strut
\end{minipage} &
\begin{minipage}[t]{(\columnwidth - 4\tabcolsep) * \real{0.14}}\centering
5\strut
\end{minipage} &
\begin{minipage}[t]{(\columnwidth - 4\tabcolsep) * \real{0.08}}\centering
19\strut
\end{minipage}\tabularnewline
\begin{minipage}[t]{(\columnwidth - 4\tabcolsep) * \real{0.19}}\centering
\textbf{helps}\strut
\end{minipage} &
\begin{minipage}[t]{(\columnwidth - 4\tabcolsep) * \real{0.10}}\centering
6\strut
\end{minipage} &
\begin{minipage}[t]{(\columnwidth - 4\tabcolsep) * \real{0.12}}\centering
14\strut
\end{minipage} &
\begin{minipage}[t]{(\columnwidth - 4\tabcolsep) * \real{0.14}}\centering
18\strut
\end{minipage} &
\begin{minipage}[t]{(\columnwidth - 4\tabcolsep) * \real{0.08}}\centering
38\strut
\end{minipage}\tabularnewline
\begin{minipage}[t]{(\columnwidth - 4\tabcolsep) * \real{0.19}}\centering
\textbf{Sum}\strut
\end{minipage} &
\begin{minipage}[t]{(\columnwidth - 4\tabcolsep) * \real{0.10}}\centering
30\strut
\end{minipage} &
\begin{minipage}[t]{(\columnwidth - 4\tabcolsep) * \real{0.12}}\centering
30\strut
\end{minipage} &
\begin{minipage}[t]{(\columnwidth - 4\tabcolsep) * \real{0.14}}\centering
30\strut
\end{minipage} &
\begin{minipage}[t]{(\columnwidth - 4\tabcolsep) * \real{0.08}}\centering
90\strut
\end{minipage}\tabularnewline
\bottomrule
\end{longtable}

\begin{center}\rule{0.5\linewidth}{0.5pt}\end{center}

\textbf{DIRECTIONS:} Use the \texttt{chisq.test()} function to perform a
two-way Chi-Squared test for independence to see if the observed counts
provide evidence of an association between the level of wealth and
reaction. Save the fitted model as \texttt{chisq\_react\_wealth}.

\begin{quote}
\textbf{NOTE:} You do not need to declare any options inside the
\texttt{chisq.test()} function, as the default is test for independence
when given a table. The \texttt{correct\ =\ FALSE} id needed only for
\(2x2\) tables.
\end{quote}

\begin{Shaded}
\begin{Highlighting}[]
\CommentTok{\# Run the 2{-}way chi{-}square test for independence}
\end{Highlighting}
\end{Shaded}

\textbf{DIRECTIONS:} Display the counts expected if reaction is
independent of wealth by starting with the model name
\texttt{chisq\_react\_wealth} and adding \texttt{\$expected} at the end.

\begin{Shaded}
\begin{Highlighting}[]
\CommentTok{\# Request the expected counts based on "no association"}
\end{Highlighting}
\end{Shaded}

\textbf{DIRECTIONS:} Place the model's name
(\texttt{chisq\_react\_wealth}) in the following chunck, so that when
run it will display the full output of the Chi-squared test.

\begin{Shaded}
\begin{Highlighting}[]
\CommentTok{\# Diplay the full output}
\end{Highlighting}
\end{Shaded}

\textbf{DIRECTIONS:} Place the data table (\texttt{react\_wealth}) into
the function \texttt{cramers\_v()} from the \texttt{effectsize} package
to get the effect size, which is called ``phi'' \(\phi\) or "Cramer's V.

\clearpage

\hypertarget{speed_voice-dichotimize-reaction-time-to-voice-calling-for-help}{%
\subsection{\texorpdfstring{\texttt{speed\_voice} Dichotimize Reaction
Time to Voice Calling for
Help}{speed\_voice Dichotimize Reaction Time to Voice Calling for Help}}\label{speed_voice-dichotimize-reaction-time-to-voice-calling-for-help}}

\hypertarget{b-8-2-way-chi-squared-test---independence}{%
\subsubsection{20B 8: 2-way Chi-squared Test -
Independence}\label{b-8-2-way-chi-squared-test---independence}}

\textbf{TEXTBOOK QUESTION:} \emph{In Exercise 12B4, the dependent
variable was the amount of time a subject listened to taperecorded cries
for help from the next room before getting up to do something. If some
subjects never respond within the time allotted for the experiment, the
validity of using parametric statistical techniques could be questioned.
As an alternative, subjects could be classified as fast or slow
responders (and possibly, nonresponders). The data from Exercise 12B4
were used to classify subjects as fast responders (less than 12 seconds
to respond) or slow responders (12 seconds or more). The resulting
contingency table is shown in the following table:}

\begin{Shaded}
\begin{Highlighting}[]
\CommentTok{\# Display the observed counts}
\NormalTok{speed\_voice }\SpecialCharTok{\%\textgreater{}\%} 
  \FunctionTok{addmargins}\NormalTok{() }\SpecialCharTok{\%\textgreater{}\%} 
\NormalTok{  pander}\SpecialCharTok{::}\FunctionTok{pander}\NormalTok{()}
\end{Highlighting}
\end{Shaded}

\begin{longtable}[]{@{}ccccc@{}}
\toprule
\begin{minipage}[b]{(\columnwidth - 4\tabcolsep) * \real{0.15}}\centering
~\strut
\end{minipage} &
\begin{minipage}[b]{(\columnwidth - 4\tabcolsep) * \real{0.11}}\centering
child\strut
\end{minipage} &
\begin{minipage}[b]{(\columnwidth - 4\tabcolsep) * \real{0.11}}\centering
woman\strut
\end{minipage} &
\begin{minipage}[b]{(\columnwidth - 4\tabcolsep) * \real{0.08}}\centering
man\strut
\end{minipage} &
\begin{minipage}[b]{(\columnwidth - 4\tabcolsep) * \real{0.08}}\centering
Sum\strut
\end{minipage}\tabularnewline
\midrule
\endhead
\begin{minipage}[t]{(\columnwidth - 4\tabcolsep) * \real{0.15}}\centering
\textbf{fast}\strut
\end{minipage} &
\begin{minipage}[t]{(\columnwidth - 4\tabcolsep) * \real{0.11}}\centering
5\strut
\end{minipage} &
\begin{minipage}[t]{(\columnwidth - 4\tabcolsep) * \real{0.11}}\centering
3\strut
\end{minipage} &
\begin{minipage}[t]{(\columnwidth - 4\tabcolsep) * \real{0.08}}\centering
1\strut
\end{minipage} &
\begin{minipage}[t]{(\columnwidth - 4\tabcolsep) * \real{0.08}}\centering
9\strut
\end{minipage}\tabularnewline
\begin{minipage}[t]{(\columnwidth - 4\tabcolsep) * \real{0.15}}\centering
\textbf{slow}\strut
\end{minipage} &
\begin{minipage}[t]{(\columnwidth - 4\tabcolsep) * \real{0.11}}\centering
2\strut
\end{minipage} &
\begin{minipage}[t]{(\columnwidth - 4\tabcolsep) * \real{0.11}}\centering
4\strut
\end{minipage} &
\begin{minipage}[t]{(\columnwidth - 4\tabcolsep) * \real{0.08}}\centering
6\strut
\end{minipage} &
\begin{minipage}[t]{(\columnwidth - 4\tabcolsep) * \real{0.08}}\centering
12\strut
\end{minipage}\tabularnewline
\begin{minipage}[t]{(\columnwidth - 4\tabcolsep) * \real{0.15}}\centering
\textbf{Sum}\strut
\end{minipage} &
\begin{minipage}[t]{(\columnwidth - 4\tabcolsep) * \real{0.11}}\centering
7\strut
\end{minipage} &
\begin{minipage}[t]{(\columnwidth - 4\tabcolsep) * \real{0.11}}\centering
7\strut
\end{minipage} &
\begin{minipage}[t]{(\columnwidth - 4\tabcolsep) * \real{0.08}}\centering
7\strut
\end{minipage} &
\begin{minipage}[t]{(\columnwidth - 4\tabcolsep) * \real{0.08}}\centering
21\strut
\end{minipage}\tabularnewline
\bottomrule
\end{longtable}

\begin{center}\rule{0.5\linewidth}{0.5pt}\end{center}

\textbf{TEXTBOOK QUESTION:} \emph{(a) Test the null hypothesis
(\(\alpha = .05\)) that speed of response is independent of type of
voice heard.}

\textbf{DIRECTIONS:} Use the \texttt{chisq.test()} function to perform a
two-way Chi-Squared test for independence to see if the observed counts
provide evidence of an association between the level of wealth and
reaction. Save the fitted model as \texttt{chisq\_speed\_voice}.

\begin{quote}
\textbf{NOTE:} You do not need to declare any options inside the
\texttt{chisq.test()} function, as the default is test for independence
when given a table. The \texttt{correct\ =\ FALSE} id needed only for
\(2x2\) tables.
\end{quote}

\begin{Shaded}
\begin{Highlighting}[]
\CommentTok{\# Run the 2{-}way chi{-}square test for independence}
\end{Highlighting}
\end{Shaded}

\textbf{DIRECTIONS:} Display the counts expected if reaction is
independent of wealth by starting with the model name
\texttt{chisq\_speed\_voice} and adding \texttt{\$expected} at the end.

\begin{Shaded}
\begin{Highlighting}[]
\CommentTok{\# Request the expected counts based on "no association"}
\end{Highlighting}
\end{Shaded}

\textbf{DIRECTIONS:} Place the model's name
(\texttt{chisq\_speed\_voice}) in the following chunck, so that when run
it will display the full output of the Chi-squared test.

\begin{Shaded}
\begin{Highlighting}[]
\CommentTok{\# Diplay the full output}
\end{Highlighting}
\end{Shaded}

\textbf{DIRECTIONS:} Place the data table (\texttt{speed\_voice}) into
the function \texttt{cramers\_v()} from the \texttt{effectsize} package
to get the effect size, which is called ``phi'' \(\phi\) or "Cramer's V.

\clearpage

\textbf{TEXTBOOK QUESTION:} \emph{(b) How does your conclusion in part a
compare with the conclusion you drew in Exercise 12B4? Categorizing the
dependent variable throws away information; how do you think that loss
of information affects power?}

\begin{Shaded}
\begin{Highlighting}[]
\NormalTok{data\_12b4 }\SpecialCharTok{\%\textgreater{}\%} 
\NormalTok{  pander}\SpecialCharTok{::}\FunctionTok{pander}\NormalTok{()}
\end{Highlighting}
\end{Shaded}

\begin{longtable}[]{@{}ccc@{}}
\toprule
\begin{minipage}[b]{(\columnwidth - 2\tabcolsep) * \real{0.11}}\centering
child\strut
\end{minipage} &
\begin{minipage}[b]{(\columnwidth - 2\tabcolsep) * \real{0.11}}\centering
woman\strut
\end{minipage} &
\begin{minipage}[b]{(\columnwidth - 2\tabcolsep) * \real{0.11}}\centering
man\strut
\end{minipage}\tabularnewline
\midrule
\endhead
\begin{minipage}[t]{(\columnwidth - 2\tabcolsep) * \real{0.11}}\centering
10\strut
\end{minipage} &
\begin{minipage}[t]{(\columnwidth - 2\tabcolsep) * \real{0.11}}\centering
17\strut
\end{minipage} &
\begin{minipage}[t]{(\columnwidth - 2\tabcolsep) * \real{0.11}}\centering
20\strut
\end{minipage}\tabularnewline
\begin{minipage}[t]{(\columnwidth - 2\tabcolsep) * \real{0.11}}\centering
12\strut
\end{minipage} &
\begin{minipage}[t]{(\columnwidth - 2\tabcolsep) * \real{0.11}}\centering
13\strut
\end{minipage} &
\begin{minipage}[t]{(\columnwidth - 2\tabcolsep) * \real{0.11}}\centering
25\strut
\end{minipage}\tabularnewline
\begin{minipage}[t]{(\columnwidth - 2\tabcolsep) * \real{0.11}}\centering
15\strut
\end{minipage} &
\begin{minipage}[t]{(\columnwidth - 2\tabcolsep) * \real{0.11}}\centering
16\strut
\end{minipage} &
\begin{minipage}[t]{(\columnwidth - 2\tabcolsep) * \real{0.11}}\centering
14\strut
\end{minipage}\tabularnewline
\begin{minipage}[t]{(\columnwidth - 2\tabcolsep) * \real{0.11}}\centering
11\strut
\end{minipage} &
\begin{minipage}[t]{(\columnwidth - 2\tabcolsep) * \real{0.11}}\centering
12\strut
\end{minipage} &
\begin{minipage}[t]{(\columnwidth - 2\tabcolsep) * \real{0.11}}\centering
17\strut
\end{minipage}\tabularnewline
\begin{minipage}[t]{(\columnwidth - 2\tabcolsep) * \real{0.11}}\centering
5\strut
\end{minipage} &
\begin{minipage}[t]{(\columnwidth - 2\tabcolsep) * \real{0.11}}\centering
7\strut
\end{minipage} &
\begin{minipage}[t]{(\columnwidth - 2\tabcolsep) * \real{0.11}}\centering
12\strut
\end{minipage}\tabularnewline
\begin{minipage}[t]{(\columnwidth - 2\tabcolsep) * \real{0.11}}\centering
7\strut
\end{minipage} &
\begin{minipage}[t]{(\columnwidth - 2\tabcolsep) * \real{0.11}}\centering
8\strut
\end{minipage} &
\begin{minipage}[t]{(\columnwidth - 2\tabcolsep) * \real{0.11}}\centering
18\strut
\end{minipage}\tabularnewline
\begin{minipage}[t]{(\columnwidth - 2\tabcolsep) * \real{0.11}}\centering
2\strut
\end{minipage} &
\begin{minipage}[t]{(\columnwidth - 2\tabcolsep) * \real{0.11}}\centering
3\strut
\end{minipage} &
\begin{minipage}[t]{(\columnwidth - 2\tabcolsep) * \real{0.11}}\centering
7\strut
\end{minipage}\tabularnewline
\bottomrule
\end{longtable}

\begin{center}\rule{0.5\linewidth}{0.5pt}\end{center}

\begin{Shaded}
\begin{Highlighting}[]
\NormalTok{data\_12b4 }\SpecialCharTok{\%\textgreater{}\%} 
\NormalTok{  tidyr}\SpecialCharTok{::}\FunctionTok{gather}\NormalTok{(}\AttributeTok{key =}\NormalTok{ voice,}
                \AttributeTok{value =}\NormalTok{ seconds,}
\NormalTok{                child, woman, man) }\SpecialCharTok{\%\textgreater{}\%} 
\NormalTok{  dplyr}\SpecialCharTok{::}\FunctionTok{mutate}\NormalTok{(}\AttributeTok{voice =} \FunctionTok{factor}\NormalTok{(voice,}
                               \AttributeTok{levels =} \FunctionTok{c}\NormalTok{(}\StringTok{"child"}\NormalTok{, }\StringTok{"woman"}\NormalTok{, }\StringTok{"man"}\NormalTok{))) }\SpecialCharTok{\%\textgreater{}\%} 
\NormalTok{  dplyr}\SpecialCharTok{::}\FunctionTok{mutate}\NormalTok{(}\AttributeTok{id =} \FunctionTok{row\_number}\NormalTok{()) }\SpecialCharTok{\%\textgreater{}\%} 
\NormalTok{  afex}\SpecialCharTok{::}\FunctionTok{aov\_4}\NormalTok{(seconds }\SpecialCharTok{\textasciitilde{}}\NormalTok{ voice }\SpecialCharTok{+}\NormalTok{ (}\DecValTok{1}\SpecialCharTok{|}\NormalTok{id),}
              \AttributeTok{data =}\NormalTok{ .) }\SpecialCharTok{\%\textgreater{}\%} 
  \FunctionTok{summary}\NormalTok{()}
\end{Highlighting}
\end{Shaded}

\begin{verbatim}
Anova Table (Type 3 tests)

Response: seconds
      num Df den Df    MSE      F     ges  Pr(>F)  
voice      2     18 26.476 3.7464 0.29392 0.04362 *
---
Signif. codes:  0 '***' 0.001 '**' 0.01 '*' 0.05 '.' 0.1 ' ' 1
\end{verbatim}

\clearpage

\hypertarget{section-c}{%
\section{Section C}\label{section-c}}

\hypertarget{import-data-define-factors-and-compute-new-variables}{%
\subsection{Import Data, Define Factors, and Compute New
Variables}\label{import-data-define-factors-and-compute-new-variables}}

Import Data, Define Factors, and Compute New Variables

\begin{itemize}
\tightlist
\item
  Make sure the \textbf{dataset} is saved in the same \emph{folder} as
  this file
\item
  Make sure the that \emph{folder} is the \textbf{working directory}
\end{itemize}

\begin{quote}
NOTE: I added the second line to convert all the variables names to
lower case. I still kept the \texttt{F} as a capital letter at the end
of the five factor variables.
\end{quote}

\begin{Shaded}
\begin{Highlighting}[]
\NormalTok{ihno\_clean }\OtherTok{\textless{}{-}} \FunctionTok{read\_excel}\NormalTok{(}\StringTok{"Ihno\_dataset.xls"}\NormalTok{) }\SpecialCharTok{\%\textgreater{}\%} 
\NormalTok{  dplyr}\SpecialCharTok{::}\FunctionTok{rename\_all}\NormalTok{(tolower) }\SpecialCharTok{\%\textgreater{}\%} 
\NormalTok{  dplyr}\SpecialCharTok{::}\FunctionTok{mutate}\NormalTok{(}\AttributeTok{genderF =} \FunctionTok{factor}\NormalTok{(gender, }
                                 \AttributeTok{levels =} \FunctionTok{c}\NormalTok{(}\DecValTok{1}\NormalTok{, }\DecValTok{2}\NormalTok{),}
                                 \AttributeTok{labels =} \FunctionTok{c}\NormalTok{(}\StringTok{"Female"}\NormalTok{, }
                                            \StringTok{"Male"}\NormalTok{))) }\SpecialCharTok{\%\textgreater{}\%} 
\NormalTok{  dplyr}\SpecialCharTok{::}\FunctionTok{mutate}\NormalTok{(}\AttributeTok{majorF =} \FunctionTok{factor}\NormalTok{(major, }
                                \AttributeTok{levels =} \FunctionTok{c}\NormalTok{(}\DecValTok{1}\NormalTok{, }\DecValTok{2}\NormalTok{, }\DecValTok{3}\NormalTok{, }\DecValTok{4}\NormalTok{,}\DecValTok{5}\NormalTok{),}
                                \AttributeTok{labels =} \FunctionTok{c}\NormalTok{(}\StringTok{"Psychology"}\NormalTok{,}
                                           \StringTok{"Premed"}\NormalTok{,}
                                           \StringTok{"Biology"}\NormalTok{,}
                                           \StringTok{"Sociology"}\NormalTok{,}
                                           \StringTok{"Economics"}\NormalTok{))) }\SpecialCharTok{\%\textgreater{}\%} 
\NormalTok{  dplyr}\SpecialCharTok{::}\FunctionTok{mutate}\NormalTok{(}\AttributeTok{reasonF =} \FunctionTok{factor}\NormalTok{(reason,}
                                 \AttributeTok{levels =} \FunctionTok{c}\NormalTok{(}\DecValTok{1}\NormalTok{, }\DecValTok{2}\NormalTok{, }\DecValTok{3}\NormalTok{),}
                                 \AttributeTok{labels =} \FunctionTok{c}\NormalTok{(}\StringTok{"Program requirement"}\NormalTok{,}
                                            \StringTok{"Personal interest"}\NormalTok{,}
                                            \StringTok{"Advisor recommendation"}\NormalTok{))) }\SpecialCharTok{\%\textgreater{}\%} 
\NormalTok{  dplyr}\SpecialCharTok{::}\FunctionTok{mutate}\NormalTok{(}\AttributeTok{exp\_condF =} \FunctionTok{factor}\NormalTok{(exp\_cond,}
                                   \AttributeTok{levels =} \FunctionTok{c}\NormalTok{(}\DecValTok{1}\NormalTok{, }\DecValTok{2}\NormalTok{, }\DecValTok{3}\NormalTok{, }\DecValTok{4}\NormalTok{),}
                                   \AttributeTok{labels =} \FunctionTok{c}\NormalTok{(}\StringTok{"Easy"}\NormalTok{,}
                                              \StringTok{"Moderate"}\NormalTok{,}
                                              \StringTok{"Difficult"}\NormalTok{,}
                                              \StringTok{"Impossible"}\NormalTok{))) }\SpecialCharTok{\%\textgreater{}\%} 
\NormalTok{  dplyr}\SpecialCharTok{::}\FunctionTok{mutate}\NormalTok{(}\AttributeTok{coffeeF =} \FunctionTok{factor}\NormalTok{(coffee,}
                                 \AttributeTok{levels =} \FunctionTok{c}\NormalTok{(}\DecValTok{0}\NormalTok{, }\DecValTok{1}\NormalTok{),}
                                 \AttributeTok{labels =} \FunctionTok{c}\NormalTok{(}\StringTok{"Not a regular coffee drinker"}\NormalTok{,}
                                            \StringTok{"Regularly drinks coffee"}\NormalTok{)))  }\SpecialCharTok{\%\textgreater{}\%} 
\NormalTok{  dplyr}\SpecialCharTok{::}\FunctionTok{mutate}\NormalTok{(}\AttributeTok{hr\_base\_bps =}\NormalTok{ hr\_base }\SpecialCharTok{/} \DecValTok{60}\NormalTok{) }
\end{Highlighting}
\end{Shaded}

\clearpage

\hypertarget{ihno_clean-ihnos-dataset}{%
\subsection{\texorpdfstring{\texttt{ihno\_clean} Ihno's
Dataset}{ihno\_clean Ihno's Dataset}}\label{ihno_clean-ihnos-dataset}}

\hypertarget{c-1a-1-way-chi-squared-test---goodness-of-fit-equally-likely}{%
\subsubsection{20C 1a: 1-way Chi-squared Test - Goodness-of-Fit (equally
likely)}\label{c-1a-1-way-chi-squared-test---goodness-of-fit-equally-likely}}

\textbf{TEXTBOOK QUESTION:} \emph{(a) Perform a one-way chi square test
to determine whether you can reject the null hypothesis that, at Ihno's
university, there are the same number of students majoring in each of
the five areas represented in Ihno's class, if you assume that Ihno's
students represent a random sample with respect to major area.}

\begin{center}\rule{0.5\linewidth}{0.5pt}\end{center}

\textbf{DIRECTIONS:} Use the \texttt{chisq.test()} function to perform a
Goodnes-of-Fit or one-way Chi-Squared test to see if the observed counts
are significantly different from being equally distributed among the
five majors. Save the fitted model as \texttt{chisq\_ihno\_major}.

\begin{quote}
\textbf{HINT:} Since you are working from a full dataset, you will need
to pipe a \texttt{dplyr::select(majorF)} step onto the
\texttt{ihno\_clean} dataset to first select out just the
\texttt{majorF} variable and then pipe on the \texttt{table()} function
to tablulate the observed counts for each major. Then and only then, you
may add the \texttt{chisq.test()} function.
\end{quote}

\begin{quote}
\textbf{NOTE:} You do not need to declare any options inside the
\texttt{chisq.test()} function, as the default is to use equally likely
probabilities.
\end{quote}

\textbf{DIRECTIONS:} Folow the tutorial to create a table comparing the
observed and expected counts.

\begin{quote}
\textbf{HINT} You may \emph{copy-and-paste} the code from the chunked
named \texttt{tutorial\_chiSq\_GoF\_EL\_counts}, but remember to change
the name of the model (appears before the \$-sign in two places).
\end{quote}

\begin{Shaded}
\begin{Highlighting}[]
\CommentTok{\# Request the observed and expected counts}
\end{Highlighting}
\end{Shaded}

\textbf{DIRECTIONS:} Place the model's name
(\texttt{chisq\_ihno\_major}) in the following chunck, so that when run
it will display the full output of the Chi-squared test.

\begin{Shaded}
\begin{Highlighting}[]
\CommentTok{\# Diplay the full output}
\end{Highlighting}
\end{Shaded}

\clearpage

\hypertarget{c-1b-repeat-separately-for-each-gender}{%
\subsubsection{20C 1b: Repeat, separately for each
gender}\label{c-1b-repeat-separately-for-each-gender}}

\textbf{TEXTBOOK QUESTION:} \emph{(b) Perform the test in part a
separately for both the males and the females in Ihno's class.}

\begin{center}\rule{0.5\linewidth}{0.5pt}\end{center}

\textbf{DIRECTIONS:} Perform the same test you did in part a, but
separately for each level of the gender variable.

\begin{quote}
\textbf{HINT} You may \emph{copy-and-paste} the code from the chunked
named \texttt{20c1a\_chiSq\_GoF\_EL\_test}, but do NOT same the model as
anything.
\end{quote}

\begin{quote}
\textbf{NOTE:} You will need to add a
\texttt{dplyr::filter(genderF\ ==\ "Male")} step before the selecting of
major.
\end{quote}

\begin{center}\rule{0.5\linewidth}{0.5pt}\end{center}

\begin{quote}
\textbf{HINT} You may \emph{copy-and-paste} the code chunk directly
above, changing only \texttt{"Male"} to \texttt{"Female"}.
\end{quote}

\clearpage

\hypertarget{c-3-2-way-chi-squared-test---independence}{%
\subsubsection{20C 3: 2-way Chi-squared Test -
Independence}\label{c-3-2-way-chi-squared-test---independence}}

\textbf{TEXTBOOK QUESTION:} \emph{Conduct a two-way chi-square analysis
of Ihno's data to test the null hypothesis that the proportion of
females is the same for each of the five represented majors in the
entire university population. \sout{Request a statistic to describe the
strength of the relationship between gender and major.}}

\begin{quote}
\textbf{NOTE:} The \texttt{furniture} package includes a very helpful
function called \texttt{tableX()} which creates a nice cross-tabulation
given the names of two variables.
\end{quote}

\begin{Shaded}
\begin{Highlighting}[]
\NormalTok{ihno\_clean }\SpecialCharTok{\%\textgreater{}\%} 
\NormalTok{  furniture}\SpecialCharTok{::}\FunctionTok{tableX}\NormalTok{(genderF, majorF)}
\end{Highlighting}
\end{Shaded}

\begin{verbatim}
        majorF
genderF  Psychology Premed Biology Sociology Economics Total
  Female 19         11     11      12        4         57   
  Male   10         14     10      3         6         43   
  Total  29         25     21      15        10        100  
\end{verbatim}

\begin{center}\rule{0.5\linewidth}{0.5pt}\end{center}

\textbf{DIRECTIONS:} Use the \texttt{chisq.test()} function to perform a
two-way Chi-Squared test for independence to see if the observed counts
are significantly different from thoes expected is there is no
association between gender and major.

\begin{quote}
\textbf{HINT:} Since you are working from a full dataset, you will need
to pipe a \texttt{dplyr::select(genderF,\ majorF)} step onto the
\texttt{ihno\_clean} dataset to first select out just the
\texttt{genderF} and \texttt{majorF} variables. Then pipe on the
\texttt{table()} function to cross-tablulate the observed counts. Then
and only then, you may add the \texttt{chisq.test()} function.
\end{quote}

\begin{quote}
\textbf{NOTE:} If you do not save the model to a name, the full output
will be displayed.
\end{quote}

\textbf{DIRECTIONS:} Follow the data table with the function
\texttt{cramers\_v()} from the \texttt{effectsize} package to get the
effect size, which is called ``phi'' \(\phi_C\) or "Cramer's V.

\end{document}
